\documentclass[a4paper]{article}
\usepackage[spanish,activeacute]{babel}
\usepackage[ansinew]{inputenc}
% Anda fen'omeno mientras codifiquemos el archivo como ansi.
\usepackage{graphicx}
\usepackage[left=2cm,right=2cm]{geometry}
\usepackage{ulem} %Para tachar cosas
%\usepackage{epigraph}
%\usepackage{listings}
\usepackage{html}
\usepackage[colorlinks=true]{hyperref}
\parindent = 0 pt
\parskip = 11 pt

\begin{document}

	\thispagestyle{empty}
	\begin{center}
	\vspace*{\stretch{2}}
	{\Large Propuesta de Tesis de Licenciatura}\\[1em]
	{\huge \textbf{$\lambda$Page}}\\[0.5em]
	{\large \textit{Un bloc de notas para desarrolladores Haskell}}\\[1em]
	\par\vspace{\stretch{1}}
	{\large Departamento de Computaci\'on}\\[0.5em]
	{\large Facultad de Ciencias Exactas y Naturales}\\[0.5em]
	{\large Universidad de Buenos Aires}
	\par\vspace{\stretch{3}}
	{\Large \textbf{Alumno}}\\[0.8em]
	{\Large Fernando Benavides (LU 470/01)} \par
	{\Large greenmellon@gmail.com} \par
	\par\vspace{\stretch{3}}
	{\Large \textbf{Directores}}\\[0.8em]
	{\Large Dr. Diego Garbervetsky} \par
	{\Large Lic. Daniel Gor'in}
	\end{center}

\section{Motivaci'on}

\paragraph{}Actualmente, gracias al 'exito de varios proyectos desarrollados con ellos, como \htmladdnormallinkfoot{CouchDB}{http://couchdb.apache.org}, \htmladdnormallinkfoot{ejabberd}{http://www.ejabberd.im} y el chat de \htmladdnormallinkfoot{Facebook}{http://www.facebook.com}, los lenguajes del paradigma funcional, como \htmladdnormallinkfoot{Haskell}{http://www.haskell.org} o \htmladdnormallinkfoot{Erlang}{http://www.erlang.org}, han ganado un renovado inter'es entre los desarrolladores.
\subparagraph{}Estos lenguajes son maduros, confiables y presentan abundantes ventajas en comparaci'on con los lenguajes tradicionales del paradigma imperativo.  Sin embargo, aquellos desarrolladores que deciden hacer el cambio de paradigma se encuentran generalmente con el problema de la falta de herramientas a las que ya est'an acostumbrados por ser habituales al desarrollar en lenguajes orientados a objetos.  En particular, en este proyecto queremos centrar nuestra atenci'on sobre aquellas herramientas que permiten realizar \textsl{debugging} a trav'es del testeo de peque'nas porciones de c'odigo.
\subparagraph{}Los desarrolladores Haskell cuentan actualmente con dos herramientas de este tipo:
\begin{description}
	\item[\htmladdnormallinkfoot{GHCi}{http://www.haskell.org/ghc/docs/latest/html/users\_guide/ghci.html}]
		La consola que provee \htmladdnormallinkfoot{GHC}{http://www.haskell.org/ghc} permite a los desarrolladores evaluar expresiones, verificar su tipo o su clase.  Cuenta tambi'en con un \htmladdnormallinkfoot{mecanismo de debugging}{http://www.haskell.org/ghc/docs/6.10-latest/html/users\_guide/ghci-debugger.html} integrado que permite realizar la evaluaci'on de expresiones paso a paso.
	\item[\htmladdnormallinkfoot{Hat}{http://www.haskell.org/hat}]
		Un herramienta para realizar seguimiento a nivel de c'odigo fuente.  A trav'es de la generaci'on de trazas de ejecuci'on, \textit{Hat} ayuda a localizar errores en los programas y es 'util para entender su funcionamiento.
\end{description}
\subparagraph{}Por otra parte, lenguajes del paradigma imperativo nos proveen ejemplos de herramientas similares, como \htmladdnormallinkfoot{Java Scrapbook Pages}{http://help.eclipse.org/help33/index.jsp?topic=/org.eclipse.jdt.doc.user/reference/ref-34.htm} para \htmladdnormallinkfoot{Java}{http://www.java.com} y \htmladdnormallinkfoot{Workspace}{http://wiki.squeak.org/squeak/1934} para \htmladdnormallinkfoot{SmallTalk}{http://www.smalltalk.org}.  Utilizando estas herramientas, los desarrolladores pueden introducir peque'nas porciones de c'odigo, ejecutarlas y luego inspeccionar y analizar los resultados obtenidos.

\section{Propuesta de Tesis}
\subsection{Objetivo}
\paragraph{}El objetivo de esta tesis es brindar a los desarrolladores \textit{Haskell} una herramienta similar al Workspace de Smalltalk que les permita trabajar con peque'nas porciones de c'odigo, dividirlas en expresiones, evaluarlas, observar su tipo y su clase.
\subparagraph{}En el esp'iritu de las herramientas provistas por la comunidad de desarrolladores Haskell, es nuestra intenci'on que \textsl{$\lambda$Page}, tal es el nombre que hemos elegido, sea desarrollada en \textit{Haskell} y se integre con \htmladdnormallinkfoot{Cabal}{http://www.haskell.org/cabal}, \htmladdnormallinkfoot{Hayoo!}{http://holumbus.fh-wedel.de/hayoo} y otras herramientas ya existentes.
\subparagraph{} \textsl{$\lambda$Page} presentar'a una interfaz visual e intuitiva y, como su
 objetivo en parte incluye el reemplazo de \textsl{GHCi} como mecanismo de evaluaci'on e  
 inspecci'on de expresiones, debe funcionar en las plataformas en las que actualmente funciona 
 esta aplicaci'on: \textsl{OSX}, \textsl{Linux} y \textsl{Windows}, al menos.
\subparagraph{}Pretendemos tambi'en que \textsl{$\lambda$Page} se diferencie de herramientas similares del mundo de los lenguajes orientados a objetos por estar hecha con y para \textit{Haskell}, un lenguaje funcional.  \textsl{$\lambda$Page} deber'a aprovechar por tanto conceptos claves del lenguaje como la evaluaci'on peresoza y la transparencia referencial.

\subsection{Plan de Trabajo}
\paragraph{B'usqueda y An'alisis de Herramientas} Se buscar'an y analizar'an las herramientas que nos permitan desarrollar \textsl{$\lambda$Page}.  En particular, se requerir'an librer'ias para conexi'on con la VM de \textit{Haskell}, parseo de c'odigo y, sobre todo, desarrollo de interfaces gr'aficas amigables.
\paragraph{Dise'no y Desarrollo de Prototipo} Se generar'a un prototipo inicial para testear la factibilidad del proyecto.
\paragraph{Generaci'on de Primera Versi'on mediante TDD} Utilizando \htmladdnormallinkfoot{Test Driven Development}{http://www.agiledata.org/essays/tdd.html} como metodolog'ia, se generar'an sucesivas mejoras sobre el prototipo hasta alcanzar una versi'on estable que provea la funcionalidad b'asica esperada.
\paragraph{Testeo y Mejoras de Usabilidad} Se alojar'a el proyecto en \textsl{HackageDB}, de modo de poder compartirlo con la comunidad de desarrolladores, testearlo y, en sucesivas iteraciones, mejorar la aplicaci'on.  A partir de este punto esperamos poder \textbf{utilizar \textsl{$\lambda$Page} para desarrollar \textsl{$\lambda$Page}}.
\paragraph{Integraci'on con \textsl{Cabal} y \textsl{Hayoo!}} Se integrar'a \textsl{$\lambda$Page} con \textsl{Cabal} permitiendo configurar la aplicaci'on en base a un paquete \textsl{Cabal} y con \textsl{Hayoo!} permitiendo buscar all'i informaci'on sobre funciones o expresiones.
\paragraph{Experimentaci'on} Se utilizar'a \textsl{$\lambda$Page} con distintos proyectos y escenarios para comprobar su utilidad y funcionamiento.
\paragraph{Conclusiones} Finalmente, se extraer'an conclusiones tanto sobre el proceso de desarrollo de la aplicaci'on como sobre la aplicaci'on y su uso en particular.

% As'i se pone c'odigo...
%\lstset{language=C++, frame=single, tabsize=2}
%\begin{lstlisting}
%#include <utility>#include <assert.h>#include <iostream>#include "float_T.h"using namespace std;float_T::float_T() {}float_T::float_T(TIPO_DATO x, unsigned int mantisa){	original= x;	mantisaReal = mantisa;	setValor(x);}TIPO_DATO float_T::getValor(){	return deMantisa;}void float_T::setValor(TIPO_DATO numero){	int mantisa = mantisaReal;	assert(! (mantisa > MAX_MANTISA) );	unsigned char	mascara[CANT_BYTES];		for( int k = CANT_BYTES - 1; k >= 0; k-- )		if (CANT_BYTES - k <= BYTES_EXP)			mascara[k] = 0xFF;		else			mascara[k] = 0x00;		int i = CANT_BYTES - BYTES_EXP - 1;	while(mantisa >= 8)	{		mascara[i] = 0xFF;		mantisa	-= 8;		i--;	}	unsigned char quito = 0x80;	while( mantisa > 0 )	{		mascara[i] += quito;		quito = quito / 2;		mantisa--;	}	for(unsigned int j = 0; j < CANT_BYTES; j++ )	{		quito				= ((unsigned char *)&numero)[j];		mascara[j]	&= quito;	}		deMantisa= *((TIPO_DATO *)&mascara);}unsigned int float_T::getMantisa(){	return mantisaReal;}TIPO_DATO float_T::getOriginal(){	return original;}void float_T::setOriginal(TIPO_DATO y){	original= y;	}
%\end{lstlisting}

% As'i se ponen ep'igrafes...
%\begin{epigraphs}
%    \qitem{bla bla bla}{Alguien}
%    \qitem{jajaja}{Alguien Chistoso}
%\end{epigraphs}
\end{document}