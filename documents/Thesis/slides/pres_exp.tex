\documentclass{beamer}
\usepackage[applemac]{inputenc}
\usepackage[spanish,activeacute]{babel}
\usepackage{amsmath}
\usepackage{amssymb}
\usepackage{latexsym}
\usepackage{listings}

%\newcommand{\speech}[1]{{\small #1}}
\newcommand{\hpage}{$\lambda Page$}
\newcommand{\haskell}{\textsl{Haskell}}
\newcommand{\speech}[1]{}
\newcommand{\G}{\mathcal{G}}
\usetheme{DC}


\title{\hpage}
\author{\textsl{Fernando~Benavides}}
	
\institute{Departamento de Computaci�n, FCEyN,Universidad de Buenos Aires.}

\date{\today}

%\beamerdefaultoverlayspecification{<+->}

\begin{document}

\begin{frame}
	\titlepage
\end{frame}

\section{Presentaci�n}
\subsection{El Orador}
\begin{frame}
	\frametitle{El Orador}
     \emph{>qui�n soy?}\\
     \begin{itemize}
       \item Fernando Benavides
	 \end{itemize}
	\pause
     \emph{>c�mo llegu� hasta aqu�?}\\
     \begin{itemize}
       \item Alumno de Computaci�n desde 2001
       \item Programador desde hace m�s de 10 a�os
       \item Programador \textsl{Funcional} desde hace 2 a�os
	\end{itemize}
\end{frame}

\subsection{El Proyecto}
\begin{frame}
     \frametitle{\hpage}
     \emph{Un bloc de notas para usuarios \haskell}\\~\\
	\pause
     \emph{Una herramienta para...}
	\begin{itemize}
		\item debuggear
		\item entender c�digo
		\item realizar \textsl{micro-testing}
	\end{itemize}
\end{frame}

\subsection{Motivaci�n}
\begin{frame}
	\frametitle{Contexto}
	Presenciamos actualmente la aparici�n de aplicaciones desarrolladas dentro del \textsl{paradigma funcional}:
		\begin{itemize}
			\item CouchDB
			\item ejabberd
			\item Chat de Facebook
		\end{itemize}
		\pause
	Desarrolladas en lenguajes como:
		\begin{itemize}
			\item Erlang
			\item Haskell
		\end{itemize}
\end{frame}
\begin{frame}
\frametitle{Necesidades}
	Para desarrollar en \haskell\, existen herramientas como:
		\begin{itemize}
			\item GHCi
			\item Hugs
			\item Hat
		\end{itemize}
	\pause
	Pero no existen herramientas como
	\begin{itemize}
		\item \textbf{Java} Scrapbook Pages
		\item Workspace de \textbf{Smalltalk}
	\end{itemize}
\end{frame}
\begin{frame}
	\frametitle{\hpage}
	\hpage\ es una herramienta similar al Workspace de Smalltalk, en tanto:
	\begin{itemize}
		\item permite al desarrollador trabajar con texto libre
		\item detecta expresiones y definiciones v�lidas
		\item permite inspeccionarlas y evaluarlas
	\end{itemize}
	\pause
	Adem�s, \hpage, brinda otras facilidades particulares para \haskell:
	\begin{itemize}
		\item Integraci�n con \textsl{Cabal} y \textsl{Hayoo!}
		\item Aprovecha \textsl{lazy evaluation} y \textsl{tipado est�tico}
		\item Presenta resultados din�micamente
	\end{itemize}
\end{frame}

\lstset{language=haskell, frame=single, tabsize=2}
\section{Caracter�sticas Principales}
\subsection{Expresiones Simples}
\begin{frame}[fragile]
\frametitle{Interpretaci�n de Expresiones}
	\hpage\ permite interpretar expresiones como:
	\begin{lstlisting}
		[1,2,3]::[Float]
	\end{lstlisting}
	\pause
	\begin{lstlisting}
		xs = [1,2,3] :: [Float]
		ys = map (+) xs
	\end{lstlisting}
	\pause
	\begin{lstlisting}
		[1,2..]
	\end{lstlisting}
	\pause
	\begin{lstlisting}
		let loop = loop in 1:loop
	\end{lstlisting}
\end{frame}
\begin{frame}
\frametitle{Presentaci�n de Resultados}
	\begin{itemize}
		\item Para expresiones inv�lidas, \hpage\ presenta el error informado por \textsl{GHC}
		\item Para expresiones sin resultado ``visible'', \hpage\ permite conocer su tipo
		\item Para expresiones infinitas, \hpage\ presenta su valor incrementalmente hasta que el usuario cancela la evaluaci�n
		\item Para expresiones que requieren c�lculos infinitos, \hpage\ permite al usuario cancelar la evaluaci�n
	\end{itemize}
\end{frame}

\subsection{Expresiones Complejas}
\begin{frame}[fragile]
\frametitle{Acciones con Efectos Colaterales}
	En \haskell\ para realizar acciones que puedan generar efectos colaterales se utiliza la m�nada \texttt{IO}.  Por ejemplo:
	\begin{lstlisting}
		readFile ``README'' :: IO String
	\end{lstlisting}
	\pause
	\texttt{IO String} no es un tipo ``visible'', por lo que el valor de la expresi�n no se podr�a mostrar.
	\pause \\
	\hpage\ toma el modelo de \textsl{GHCi} y ejecuta la acci�n, presentando su resultado
\end{frame}
\begin{frame}[fragile]
\frametitle{Listas}
	\hpage\ interpreta de modo particular las listas (expresiones de tipo \texttt{Show a => [a]})\\
	Por ejemplo, al evaluar la siguiente expresi�n:
	\begin{lstlisting}
		let loop = loop in [1, div 0 0, 2,
		                    undefined, 3, loop, 4]
	\end{lstlisting}
	\pause
	\hpage\ podr�a presentar como resultado
	\begin{center}
	[1,
	\end{center}
	\pause
	o, en caso de detectar excepciones en los elementos de la lista,
	\begin{center}
	[1, $_{\bot}$, 2, $_{\bot}$, 3, 
	\end{center}
\end{frame}
\begin{frame}[fragile]
\frametitle{Listas}
	\hpage\ interpreta de modo particular las listas (expresiones de tipo \texttt{Show a => [a]})\\
	Por ejemplo, al evaluar la siguiente expresi�n:
	\begin{lstlisting}
		let loop = loop in [1, div 0 0, 2,
		                    undefined, 3, loop, 4]
	\end{lstlisting}
	\hpage\ presenta como resultado:
	\begin{center}
	[1, $_{\bot}$, 2, $_{\bot}$, 3, $_{\bot}$, 4]
	\end{center}
\end{frame}

\subsection{Otras Caracter�sticas}
\begin{frame}
\frametitle{Otras Caracter�sticas}
	TODO: Paralelismo\\
	TODO: Hablar de la integraci�n con Cabal y Hayoo!\\
	TODO: Hablar de Importar/Cargar/Recargar M�dulos\\
	TODO: Hablar de G�nero
\end{frame}

\section{\hpage\ por Dentro}
\subsection{Arquitectura}
\begin{frame}
\frametitle{Arquitectura}
	TODO: Gr�fico de la arquitectura y su explicaci�n
\end{frame}
\begin{frame}
\frametitle{Secuencia}
	TODO: Gr�fico de la secuencia y su explicaci�n?
\end{frame}

\subsection{Dise�o}
\begin{frame}
\frametitle{Principales Decisiones de Dise�o}
	TODO: Principales temas de dise�o
\end{frame}

\subsection{Implementaci�n}
\begin{frame}
\frametitle{Implementaci�n}
	TODO: Principales temas de implementaci�n
\end{frame}


\section{Resultados}
\subsection{Logros}
\begin{frame}
\frametitle{Objetivos Alcanzados}
	TODO: Objetivos Alcanzados
\end{frame}
\begin{frame}
\frametitle{Logros Adicionales}
	TODO: Otras cosas logradas m�s all� de lo que originalmente nos propusimos
\end{frame}

\subsection{Trabajo a Futuro}
\begin{frame}
\frametitle{Tareas a Realizar}
	TODO: Trabajo a Futuro
\end{frame}

\subsection{Agradecimientos / Preguntas}
\begin{frame}
\frametitle{!`Gracias a todos!}
	\emph{?`En qu� los puedo ayudar?}
\end{frame}
\end{document}